% !TeX root = ../main.tex

\chapter{Unique-Spike Event Analysis}

We further deep dive into unique spikes from our index and their underlying causes. Table~\ref{tab:unique-spikes} summarizes each spike. ALL DESCRIPTION NEED FULL REWRITE.
The summary show that our index unique strength is capturing hybrid-warfare events including diplomatic isolation (Solomon Islands, Kiribati), economic uncertainty(trade-war), foreign intervention potential swing(US election).


\begin{table}[htbp]
\small                               % switch to \footnotesize if still too wide
\caption{Major unique spikes in the GPR index}
\label{tab:unique-spikes}
\centering
%            date        event          description (auto-wrap)        keywords
\begin{tabularx}{\textwidth}{@{}L{2.4cm}L{3.1cm}X L{3.5cm}@{}}
\toprule
Peak date & Event & Description & keywords\\
\midrule
2018-05-21 & US–China trade war &
The GPR spike on May 21 2018 was driven by intensifying US-China trade-war negotiations, North Korea’s invitation to journalists for its nuclear-site dismantling, and a US embassy-staff injury that raised security concerns. &
美中貿易戰, 川金會, 北韓核試場拆除, 台海關係, 地緣政治\\[4pt]

2018-12-31 & Cross-Strait tensions &
President Tsai’s “four musts” speech, Xi Jinping’s “one country, two systems” remarks, plus US-China trade-war friction and African-swine-fever worries all pushed the index higher. &
兩岸關係, 美中貿易戰, 非洲豬瘟, 跨年活動\\[4pt]

2019-04-01 & Cross-Strait military provocation &
Chinese fighter jets crossed the Taiwan-Strait median line amid debate over US influence and Taiwanese domestic politics, sharply raising perceived risk. &
中國軍機越線, 台美關係, 兩岸關係, 政治角力\\[4pt]

2019-09-16 & Middle-East tensions &
The Saudi-oil-facility attack spiked global oil prices; Hong-Kong protests and Taiwan’s diplomatic losses (Solomon Islands, Kiribati) added to uncertainty. &
沙烏地油廠遇襲, 香港反送中, 美中貿易戰, 台灣邦交國斷交\\[4pt]

2020-01-06 & Middle-East tensions &
After the US strike on Qasem Soleimani, US-Iran tensions surged while early Wuhan “unknown pneumonia” reports emerged, amplifying global risk. &
不明原因肺炎, 武漢, 中東局勢, 美伊衝突, 選舉\\[4pt]

2020-11-02 & US-election uncertainty &
Heightened geopolitical risk stemmed from the tight US presidential race, ongoing pandemic worries, and implications for US-China–Taiwan relations. &
美國大選, 兩岸關係, 新冠疫情, 經濟前景\\[4pt]

2021-12-06 & US–China Olympic boycott &
The US announced a diplomatic boycott of the Beijing Winter Olympics; tensions over Ukraine and democracy-alliance diplomacy (e.g., Taiwan in the Summit for Democracy) added momentum. &
北京冬奧抵制, 台美關係, 烏克蘭危機, 中國人權\\[4pt]

2024-08-26 & US-election tensions &
Donald Trump’s rhetoric on trade, the ongoing Russia-Ukraine war, cross-strait friction, and global inflation fears combined to lift the index. &
川普, 俄烏戰爭, 兩岸關係, 美國大選, 經濟\\
\bottomrule
\end{tabularx}
\end{table}
