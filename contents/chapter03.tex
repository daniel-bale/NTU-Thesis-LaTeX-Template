% !TeX root = ../main.tex

\chapter{Data \& Dictionary Construction}

This chapter describes the data sources and dictionary creation process.

\begin{table}[htbp]
\centering
\caption{Table 1: Search Query for the Taiwan Cross-Strait GPR Index}
\begin{tabular}{llll}
\toprule
Category & Search query & Peak (week) & Contribution to index percent \\
\midrule
\multicolumn{4}{l}{\textbf{Panel A. Search categories and search queries}} \\
Military Escalation & military\_threats AND cross\_strait\_relations & & \\
Foreign Intervention Risk & foreign\_intervention AND military\_threats & & \\
Economic Coercion & economic\_coercion AND cross\_strait\_relations & & \\
Cyber and Information Warfare & cyber\_and\_information\_warfare AND cross\_strait\_relations & & \\
Diplomatic Crisis & diplomatic\_incidents AND foreign\_intervention & & \\
Internal Political Instability & internal\_political\_risk AND cross\_strait\_relations & & \\
Critical Infrastructure Threat & critical\_infrastructure\_threats AND military\_threats & & \\
\midrule
\multicolumn{4}{l}{\textbf{Panel B. Search word sets}} \\
\textit{military\_threats} &
\multicolumn{3}{l}{%
\begin{tabular}[t]{@{}l@{}}
軍事威脅, 武力犯台, 軍演, 軍事演習, 共軍, 解放軍, 軍機擾台, 軍艦繞台,\\
飛彈試射, 戰備警戒, 防空識別區, 空中攔截, 軍事衝突, 台海危機, 軍事部署, 戰爭邊緣
\end{tabular}
} \\
\textit{cross\_strait\_relations} &
\multicolumn{3}{l}{%
\begin{tabular}[t]{@{}l@{}}
兩岸關係, 台海局勢, 台海緊張, 和平協議, 九二共識, 一國兩制, 統一, 分裂, 台獨,\\
獨立公投, 和平發展, 敵對狀態, 兩岸對話, 兩岸交流
\end{tabular}
} \\
\textit{foreign\_intervention} &
\multicolumn{3}{l}{%
\begin{tabular}[t]{@{}l@{}}
美國介入, 美台關係, 美國軍售, 美國國會訪台, 美國支持, 日台合作, 日美安保,\\
國際聲援, 外國干預, 外國勢力, 國際制裁, 聯合國, 外交承認, 邦交國
\end{tabular}
} \\
\textit{economic\_coercion} &
\multicolumn{3}{l}{%
\begin{tabular}[t]{@{}l@{}}
經濟制裁, 貿易封鎖, 禁運, 斷交, 出口限制, 進口禁令, 經濟脅迫,\\
產業鏈轉移, 投資審查, 金融制裁, 資本外流, 供應鏈中斷
\end{tabular}
} \\
\textit{cyber\_and\_information\_warfare} &
\multicolumn{3}{l}{%
\begin{tabular}[t]{@{}l@{}}
網路攻擊, 駭客入侵, 資訊戰, 假消息, 認知作戰, 網軍, 輿論操控,\\
網路滲透, 資安威脅, 假新聞, 網路戰爭
\end{tabular}
} \\
\textit{diplomatic\_incidents} &
\multicolumn{3}{l}{%
\begin{tabular}[t]{@{}l@{}}
外交衝突, 召回大使, 抗議, 譴責, 外交斷交, 外交壓力, 國際孤立,\\
外交爭端, 外交抗議, 外交危機
\end{tabular}
} \\
\textit{internal\_political\_risk} &
\multicolumn{3}{l}{%
\begin{tabular}[t]{@{}l@{}}
政黨輪替, 總統大選, 罷免, 立法院衝突, 社會動盪, 抗議活動,\\
群眾運動, 政治分歧, 內部矛盾, 政變, 政局不穩
\end{tabular}
} \\
\textit{critical\_infrastructure\_threats} &
\multicolumn{3}{l}{%
\begin{tabular}[t]{@{}l@{}}
關鍵基礎設施, 電力中斷, 通訊癱瘓, 能源危機, 交通癱瘓, 水資源短缺,\\
基礎設施攻擊, 供電中斷, 網路癱瘓
\end{tabular}
} \\
\bottomrule
\end{tabular}
\end{table}

The language model dictionary captures a much broader set of hybrid warfare
concepts than the manually crafted dictionary. Six of the seven categories do
not belong to traditional warfare risks. This difference indicates that the
model reflects local and recent trends. Taiwan's objective is to deter a
possible invasion from China, while China seeks to counter such preparations.
Thus, terms related to economic coercion, potential foreign intervention,
diplomatic isolation, infrastructure threats, and cyber and information
attacks are relevant. Words describing direct war acts appear less frequently
because no large-scale conflict has occurred in decades, making a standard
global dictionary such as \citet{caldara2013geopoliticalrisk} less suitable.
The language model also yields a more balanced set of categories, with the top
contributor accounting for only 33\% compared with Lau's 86\%.

